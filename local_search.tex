\section{Применение локального поиска для ускорения сходимости АГП}
Методы локального поиска могут применяться в сочетании с глобальными алгоритмами для улучшения полученных решений или текущих оценок оптимума.
В первом случае локальный метод стартует из точки, найденной глобальным методом, и уточняет решение практически до любой нужной точности. Это
позволяет избежать чрезмерных затрат на поиск решения с высокой точностью глобальным методам.
\par
Во втором случае локальный метод используется для ускорения обнаружения локальных оптимумов.
Информацилнно-статистический метод Стронгина позволяет обновлять свою поисковую информацию из любых посторонних источников, в том числе и точки испытаний,
полученные от локального метода.
Как только глобальный метод находит новую оценку оптимума, из этой точки стартует локальный метод и все или часть испытаний, проведённых им
добавляется в поисковую информацию, далее глобальный метод продолжает работу. Каких-либо теоретических исследований подобной схемы не проводилось, поэтому её эффективность проверялась эеспериментально.
\par
В качестве метода локальной оптимизации был выбран метод Хука-Дживса \cite{himmelblau}. Он прост в реализации и для его работы не требуется знать значений
каких-либо производных оптимизируемой функции.
\par
Были проведены две серии экспериментов, соответствующих следующим схемам добавления точек, полученных локальным методом в поисковую информацию:
\begin{itemize}
		\item Добавление единственной точки, к которой сошёлся локальный метод.
    \item Добавление всех промежуточных точек.
\end{itemize}
Эксперименты проводились на классе GKLS 4d Simple, параметры метода были заданы такие же, как в разделе \ref{sec:multilev_maps}
