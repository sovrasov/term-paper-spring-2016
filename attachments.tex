\section{Приложения}
\subsection{Приложение 1}
\label{attach1}

\begin{lstlisting}[frame=single]
#ifndef __LOCALMETHOD_H__
#define __LOCALMETHOD_H__

#include "parameters.h"
#include "task.h"
#include "data.h"
#include "common.h"
#include <vector>

#define MAX_LOCAL_TRIALS_NUMBER 10000

class TLocalMethod
{
protected:

	int mDimension;
	int mConstraintsNumber;
	int mTrialsCounter;
	int mMaxTrial;

	TTrial mBestPoint;
	std::vector<TTrial> mSearchSequence;
	TTask* mPTask;

	bool mIsLogPoints;

	double mEps;
	double mStep;
	double mStepMultiplier;

	OBJECTIV_TYPE *mFunctionsArgument;
	OBJECTIV_TYPE *mStartPoint;
	OBJECTIV_TYPE* mCurrentPoint;
	OBJECTIV_TYPE* mCurrentResearchDirection;
	OBJECTIV_TYPE* mPreviousResearchDirection;

	double MakeResearch(OBJECTIV_TYPE*);
	void DoStep();
	double EvaluateObjectiveFunctiuon(const OBJECTIV_TYPE*);

public:

	TLocalMethod();
	TLocalMethod(TParameters _params, TTask* _pTask, TTrial _startPoint, bool logPoints = false);
	~TLocalMethod();

	void SetEps(double);
	void SetInitialStep(double);
	void SetStepMultiplier(double);
	void SetMaxTrials(int);

	int GetTrialsCounter() const;
	std::vector<TTrial> GetSearchSequence() const;

	TTrial StartOptimization();
};

#endif //__LOCALMETHOD_H__
\end{lstlisting}
\begin{lstlisting}[frame=single]
#include "local_method.h"
#include <cmath>
#include <cstring>
#include <algorithm>

TLocalMethod::TLocalMethod(TParameters _params, TTask* _pTask, TTrial _startPoint, bool logPoints)
{
	mEps = _params.Epsilon / 100;
	if (mEps > 0.0001)
		mEps = 0.0001;
	mBestPoint = _startPoint;
	mStep = _params.Epsilon * 2;
	mStepMultiplier = 2;
	mTrialsCounter = 0;
	mIsLogPoints = logPoints;

	mPTask = _pTask;

	mDimension = mPTask->GetN() - mPTask->GetFixedN();
	mConstraintsNumber = mPTask->GetNumOfFunc() - 1;

	mStartPoint = new OBJECTIV_TYPE[mDimension];
	std::memcpy(mStartPoint,
		_startPoint.y + mPTask->GetFixedN(),
		mDimension * sizeof(OBJECTIV_TYPE));

	mFunctionsArgument = new OBJECTIV_TYPE[mPTask->GetN()];
	std::memcpy(mFunctionsArgument,
		_startPoint.y,
		mPTask->GetN() * sizeof(OBJECTIV_TYPE));
	mMaxTrial = MAX_LOCAL_TRIALS_NUMBER;
}

TLocalMethod::TLocalMethod() : mPTask(NULL), mStartPoint(NULL),
mFunctionsArgument(NULL), mMaxTrial(MAX_LOCAL_TRIALS_NUMBER)
{
}

TLocalMethod::~TLocalMethod()
{
	if (mStartPoint)
		delete[] mStartPoint;
	if (mFunctionsArgument)
		delete[] mFunctionsArgument;
}

void TLocalMethod::DoStep()
{
	for (int i = 0; i < mDimension; i++)
		mCurrentPoint[i] = (1 + mStepMultiplier)*mCurrentResearchDirection[i] -
		mStepMultiplier*mPreviousResearchDirection[i];
}

TTrial TLocalMethod::StartOptimization()
{
	int k = 0, i = 0;
	bool needRestart = true;
	double currentFValue, nextFValue;
	mTrialsCounter = 0;

	mCurrentPoint = new OBJECTIV_TYPE[mDimension];
	mCurrentResearchDirection = new OBJECTIV_TYPE[mDimension];
	mPreviousResearchDirection = new OBJECTIV_TYPE[mDimension];

	while (mTrialsCounter < mMaxTrial) {
		i++;
		if (needRestart) {
			k = 0;
			std::memcpy(mCurrentPoint, mStartPoint, sizeof(OBJECTIV_TYPE)*mDimension);
			std::memcpy(mCurrentResearchDirection, mStartPoint, sizeof(OBJECTIV_TYPE)*mDimension);
			currentFValue = EvaluateObjectiveFunctiuon(mCurrentPoint);
			needRestart = false;
		}

		std::swap(mPreviousResearchDirection, mCurrentResearchDirection);
		std::memcpy(mCurrentResearchDirection, mCurrentPoint, sizeof(OBJECTIV_TYPE)*mDimension);
		nextFValue = MakeResearch(mCurrentResearchDirection);

		if (currentFValue > nextFValue) {
			DoStep();

			if (mIsLogPoints)
			{
				TTrial currentTrial;
				currentTrial.index = mBestPoint.index;
				currentTrial.FuncValues[currentTrial.index] = nextFValue;
				std::memcpy(currentTrial.y, mFunctionsArgument, sizeof(OBJECTIV_TYPE)*mPTask->GetFixedN());
				std::memcpy(currentTrial.y + mPTask->GetFixedN(), mCurrentPoint, sizeof(OBJECTIV_TYPE)*mDimension);
				mSearchSequence.push_back(currentTrial);
			}
			k++;
			currentFValue = nextFValue;
		}
		else if (mStep > mEps) {
			if (k != 0)
				std::memcpy(mStartPoint, mPreviousResearchDirection, sizeof(OBJECTIV_TYPE)*mDimension);
			else
				mStep /= mStepMultiplier;
			needRestart = true;
		}
		else
			break;
	}

	if (currentFValue < mBestPoint.FuncValues[mConstraintsNumber])
	{
		std::memcpy(mBestPoint.y + mPTask->GetFixedN(),
			mPreviousResearchDirection, sizeof(OBJECTIV_TYPE)*mDimension);
		mBestPoint.FuncValues[mConstraintsNumber] = currentFValue;
		mSearchSequence.push_back(mBestPoint);
	}

	delete[] mCurrentPoint;
	delete[] mPreviousResearchDirection;
	delete[] mCurrentResearchDirection;

	return mBestPoint;
}

double TLocalMethod::EvaluateObjectiveFunctiuon(const OBJECTIV_TYPE* x)
{
	if (mTrialsCounter >= mMaxTrial)
		return HUGE_VAL;

	for (int i = 0; i < mDimension; i++)
		if (x[i] < mPTask->GetA()[mPTask->GetFixedN() + i] ||
			x[i] > mPTask->GetB()[mPTask->GetFixedN() + i])
			return HUGE_VAL;

	std::memcpy(mFunctionsArgument + mPTask->GetFixedN(), x, mDimension * sizeof(OBJECTIV_TYPE));
	for (int i = 0; i <= mConstraintsNumber; i++)
	{
		double value = mPTask->GetFuncs()[i](mFunctionsArgument);
		if (i < mConstraintsNumber && value > 0)
		{
			mTrialsCounter++;
			return HUGE_VAL;
		}
		else if (i == mConstraintsNumber)
		{
			mTrialsCounter++;
			return value;
		}
	}

	return HUGE_VAL;
}

void TLocalMethod::SetEps(double eps)
{
	mEps = eps;
}

void TLocalMethod::SetInitialStep(double value)
{
	mStep = value;
}

void TLocalMethod::SetStepMultiplier(double value)
{
	mStepMultiplier = value;
}

void TLocalMethod::SetMaxTrials(int count)
{
	mMaxTrial = std::min(count, MAX_LOCAL_TRIALS_NUMBER);
}

int TLocalMethod::GetTrialsCounter() const
{
	return mTrialsCounter;
}

std::vector<TTrial> TLocalMethod::GetSearchSequence() const
{
	return mSearchSequence;
}

double TLocalMethod::MakeResearch(OBJECTIV_TYPE* startPoint)
{
	double bestValue = EvaluateObjectiveFunctiuon(startPoint);

	for (int i = 0; i < mDimension; i++)
	{
		startPoint[i] += mStep;
		double rightFvalue = EvaluateObjectiveFunctiuon(startPoint);

		if (rightFvalue > bestValue)
		{
			startPoint[i] -= 2 * mStep;
			double leftFValue = EvaluateObjectiveFunctiuon(startPoint);
			if (leftFValue > bestValue)
				startPoint[i] += mStep;
			else
				bestValue = leftFValue;
		}
		else
			bestValue = rightFvalue;
	}

	return bestValue;
}
\end{lstlisting}

\subsection{Приложение 2}
\label{attach2}
\begin{lstlisting}[frame=single]
#ifndef __DUAL_QUEUE_H__
#define __DUAL_QUEUE_H__

#include "minmaxheap.h"
#include "common.h"
#include "queue_common.h"

class TPriorityDualQueue : public TPriorityQueueCommon
{
protected:
	int MaxSize;
	int CurLocalSize;
	int CurGlobalSize;

	MinMaxHeap< TQueueElement, _less >* pGlobalHeap;
	MinMaxHeap< TQueueElement, _less >* pLocalHeap;

	void DeleteMinLocalElem();
	void DeleteMinGlobalElem();
	void ClearLocal();
	void ClearGlobal();
public:

	TPriorityDualQueue(int _MaxSize = DefaultQueueSize); // _MaxSize must be qual to 2^k - 1
	~TPriorityDualQueue();

	int GetLocalSize() const;
	int GetSize() const;
	int GetMaxSize() const;
	bool IsLocalEmpty() const;
	bool IsLocalFull() const;
	bool IsEmpty() const;
	bool IsFull() const;

	void Push(double globalKey, double localKey, void *value);
	void PushWithPriority(double globalKey, double localKey, void *value);
	void Pop(double *key, void **value);

	void PopFromLocal(double *key, void **value);

	void Clear();
	void Resize(int size);
};
#endif
\end{lstlisting}

\begin{lstlisting}[frame=single]
#include "dual_queue.h"

TPriorityDualQueue::TPriorityDualQueue(int _MaxSize)
{
	MaxSize = _MaxSize;
	CurGlobalSize = CurLocalSize = 0;
	pLocalHeap = new MinMaxHeap< TQueueElement, _less >(MaxSize);
	pGlobalHeap = new MinMaxHeap< TQueueElement, _less >(MaxSize);
}

TPriorityDualQueue::~TPriorityDualQueue()
{
	delete pLocalHeap;
	delete pGlobalHeap;
}

void TPriorityDualQueue::DeleteMinLocalElem()
{
	TQueueElement tmp = pLocalHeap->popMin();
	CurLocalSize--;

	//update linked element in the global queue
	if (tmp.pLinkedElement != NULL)
		tmp.pLinkedElement->pLinkedElement = NULL;
}

void TPriorityDualQueue::DeleteMinGlobalElem()
{
	TQueueElement tmp = pGlobalHeap->popMin();
	CurGlobalSize--;

	//update linked element in the local queue
	if (tmp.pLinkedElement != NULL)
		tmp.pLinkedElement->pLinkedElement = NULL;
}

int TPriorityDualQueue::GetLocalSize() const
{
	return CurLocalSize;
}

int TPriorityDualQueue::GetSize() const
{
	return CurGlobalSize;
}

int TPriorityDualQueue::GetMaxSize() const
{
	return MaxSize;
}

bool TPriorityDualQueue::IsLocalEmpty() const
{
	return CurLocalSize == 0;
}

bool TPriorityDualQueue::IsLocalFull() const
{
	return CurLocalSize == MaxSize;
}

bool TPriorityDualQueue::IsEmpty() const
{
	return CurGlobalSize == 0;
}

bool TPriorityDualQueue::IsFull() const
{
	return CurGlobalSize == MaxSize;
}

void TPriorityDualQueue::Push(double globalKey, double localKey, void * value)
{
	TQueueElement* pGlobalElem = NULL, *pLocalElem = NULL;
	//push to a global queue
	if (!IsFull()) {
		CurGlobalSize++;
		pGlobalElem = pGlobalHeap->push(TQueueElement(globalKey, value));
	}
	else {
		if (globalKey > pGlobalHeap->findMin().Key) {
			DeleteMinGlobalElem();
			CurGlobalSize++;
			pGlobalElem = pGlobalHeap->push(TQueueElement(globalKey, value));
		}
	}
	//push to a local queue
	if (!IsLocalFull()) {
		CurLocalSize++;
		pLocalElem = pLocalHeap->push(TQueueElement(localKey, value));
	}
	else {
		if (localKey > pLocalHeap->findMin().Key) {
			DeleteMinLocalElem();
			CurLocalSize++;
			pLocalElem = pLocalHeap->push(TQueueElement(localKey, value));
		}
	}
	//link elements
	if (pGlobalElem != NULL && pLocalElem != NULL) {
		pGlobalElem->pLinkedElement = pLocalElem;
		pLocalElem->pLinkedElement = pGlobalElem;
	}
}

void TPriorityDualQueue::PushWithPriority(double globalKey, double localKey, void * value)
{
	TQueueElement* pGlobalElem = NULL, *pLocalElem = NULL;
	//push to a global queue
	if (!IsEmpty()) {
		if (globalKey > pGlobalHeap->findMin().Key) {
			if (IsFull())
				DeleteMinGlobalElem();
			CurGlobalSize++;
			pGlobalElem = pGlobalHeap->push(TQueueElement(globalKey, value));
		}
	}
	else {
		CurGlobalSize++;
		pGlobalElem = pGlobalHeap->push(TQueueElement(globalKey, value));
	}
	//push to a local queue
	if (!IsLocalEmpty()) {
		if (localKey > pLocalHeap->findMin().Key) {
			if (IsLocalFull())
				DeleteMinLocalElem();
			CurLocalSize++;
			pLocalElem = pLocalHeap->push(TQueueElement(localKey, value));
		}
	}
	else {
		CurLocalSize++;
		pLocalElem = pLocalHeap->push(TQueueElement(localKey, value));
	}
	//link elements
	if (pGlobalElem != NULL && pLocalElem != NULL) {
		pGlobalElem->pLinkedElement = pLocalElem;
		pLocalElem->pLinkedElement = pGlobalElem;
	}
}

void TPriorityDualQueue::PopFromLocal(double * key, void ** value)
{
	TQueueElement tmp = pLocalHeap->popMax();
	*key = tmp.Key;
	*value = tmp.pValue;
	CurLocalSize--;

	//delete linked element from the global queue
	if (tmp.pLinkedElement != NULL) {
		pGlobalHeap->deleteElement(tmp.pLinkedElement);
		CurGlobalSize--;
	}
}

void TPriorityDualQueue::Pop(double * key, void ** value)
{
	TQueueElement tmp = pGlobalHeap->popMax();
	*key = tmp.Key;
	*value = tmp.pValue;
	CurGlobalSize--;

	//delete linked element from the local queue
	if (tmp.pLinkedElement != NULL)	{
		pLocalHeap->deleteElement(tmp.pLinkedElement);
		CurLocalSize--;
	}
}

void TPriorityDualQueue::Clear()
{
	ClearLocal();
	ClearGlobal();
}

void TPriorityDualQueue::Resize(int size)
{
	MaxSize = size;
	CurGlobalSize = CurLocalSize = 0;
	delete pLocalHeap;
	delete pGlobalHeap;
	pLocalHeap = new MinMaxHeap< TQueueElement, _less >(MaxSize);
	pGlobalHeap = new MinMaxHeap< TQueueElement, _less >(MaxSize);
}

void TPriorityDualQueue::ClearLocal()
{
	pLocalHeap->clear();
	CurLocalSize = 0;
}

void TPriorityDualQueue::ClearGlobal()
{
	pGlobalHeap->clear();
	CurGlobalSize = 0;
}
\end{lstlisting}
