\section{Многоуровневая схема редукции размерности с помощью разверток}
Одна из постановок задачи глобальной оптимизации звучит следующим образом: найти глобальный минимум \(N\)-мерной функции \(\phi(y)\) в гиперинтервале \(D=\{y\in R^N:a_i\leqslant x_i\leqslant{b_i}, 1\leqslant{i}\leqslant{N}\}\).
Для построения оценки глобального минимума по конечному количеству вычислений значения функции требуется, чтобы \(\phi(y)\) удовлетворяла условию Липшица.
\begin{equation}
\label{task}
\phi(y^*)=\min\{\phi(y):y\in D\}
\end{equation}
\begin{equation}
\label{lip}
|\phi(y_1)-\phi(y_2)|\leqslant L\Vert y_1-y_2\Vert,y_1,y_2\in D,0<L<\infty
\end{equation}
\par
Классической схемой редукции размерности для алгоритмов глобальной оптимизации является использование разверток --- кривых, заполняющих пространство \cite{strSergOptBook}.
\begin{displaymath}
\label{cube}
\lbrace y\in R^N:-2^{-1}\leqslant y_i\leqslant 2^{-1},1\leqslant i\leqslant N\rbrace=\{y(x):0\leqslant x\leqslant 1\}
\end{displaymath}
\par
 Такое отображение позволяет свести задачу в многомерном пространстве к решению одномерной ценой ухудшения её свойств. В частности, одномерная функция \(\phi(y(x))\) является не
 Липшицевой, а Гёльдеровой:
 \begin{displaymath}
\label{holder}
|\phi(y(x_1))-\phi(y(x_2))|\leqslant H{|x_1-x_2|}^{\frac{1}{N}},x_1,x_2\in[0,1]
\end{displaymath}
где константа Гельдера \(H\) связана с константой Липшица \(L\) соотношением
\begin{displaymath}
H=4Ld\sqrt{N},d=\max\{b_i-a_i:1\leqslant i\leqslant N\}
\end{displaymath}
\par
Теоретически с помощью этой схемы можно решить задучю любой размерности, однако на ЭВМ развертка строится с помощью конечноразрядной арифметики, из-за чего начиная с некоторого \(N^*\)
построение разветки невозможно (значение \(N^*\) зависит от максимального количества значащих разрядов в арифметике с плавающей точкой).
Понять почему это происходит нетрудно, обратившись, например к \cite{strSergOptBook}.
\par
Чтобы преодолеть эту проблему профессором В. П. Гергелем была предложена следующая идея: использовать композицию разверток мненьшей размерности для построения отображения
\(z(x): [0;1] \rightarrow D \in R^N\).
Поясним эту схему на примере редукции размерности в четырёхмерной задаче. Пусть \(y_2(x)\) --- двухмертная развертка (отображает отрезок в прямоугольник), тогда рассмотрим функцию
\(\psi(x_1,x_2)=\phi(y_2(x_1), y_2(x_2))\). К \(\psi(x_1,x_2)\) можно также применить редукцию размерности с помощью развертки. Таким образом, задав точку \(x^*\in [0;1]\),
вычислив \(y_2(x^*)=(x_1,x_2)\) и пару векторов \((y_2(x_1), y_2(x_2))\), получим четырёхмерную точку. Из инъективности \(y_2(x)\) следует инъективность \(z(x)\).
\par
Проблемой этого метода является выяснение свойств функции \(\phi(z(x))\) и возможности использования одномерного метода Стронгина с гёльдеровой метрикой для её оптимизации.
Чтобы не тратить время на теоретическое исследование, были проведены численные эксперименты с целью оценить возможности применения многоуровневой развёртки в четырёхмерном случае.
